\begin{table}[ht]
\caption{European Survey of Professional Forecasters combination results}
\label{tab:euroForecat}
\centerline{
\begin{tabular}{ccccc}
  \hline
  \hline
  Combination technique  & \multicolumn{2}{c}{HICP} & \multicolumn{2}{c}{Real GDP}\\
  \cmidrule(lr){2-3}\cmidrule(lr){4-5}
   & Relative MSE & Diebold-Mariano  & Relative MSE & Diebold-Mariano   \\
  \hline
    Mean Forecast & 1.00 & 1.00 & 1.00 & 1.00 \\
    Median Forecast & 1.00& 0.21  & 1.00  & 0.05$^\circ$  \\
    Lasso & 2.05 & 0.99 & 1.64  & 1.00\\ 
    peLasso & 2.27  & 1.00& 2.09  & 0.98  \\ 
    N1 & 0.81  & 0.04$^\circ$ & 0.88  & 0.15 \\ 
    N2 & 0.83 & 0.04$^\circ$ & 0.90 & 0.15 \\ 
    N3 & 0.82  & 0.03$^\circ$ & 0.91 & 0.14  \\  
    N4 & 0.83  & 0.02$^\circ$  & 0.91  & 0.13  \\ 
    Random Forest & 1.02  & 0.27  & 1.06  & 0.38  \\ 
    Boosted Tree & 0.97 & 0.13  & 0.96 & 0.21  \\
   \hline
   \hline
\end{tabular}}
\vspace{10pt}
\caption*{Notes: columns (2) and (4) report the RMSE of the combination method specified in column (1) divided by the RMSE the average survey forecast. A ratio less than one denotes the combination outperforming the uniform forecast. Columns (3) and (5) report the \cite{DM1995} p-value with the null hypothesis that the mean forecast is better than the model. $^\odot$, $^\circ$, $^\star$ denote DM statistics significant at the ten-percent, five-percent, and one-percent confidence level, respectively.}
\end{table}
